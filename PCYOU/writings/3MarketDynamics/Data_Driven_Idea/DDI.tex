


\documentclass[journal,12pt,onecolumn,draftclsnofoot]{IEEEtran}


\usepackage{amsthm}
\usepackage{amsmath}
\usepackage{amssymb}
\usepackage{booktabs}
\usepackage{color}
\usepackage{epsfig}
\usepackage{subfigure}
\usepackage{graphicx}
%\usepackage{caption}
\usepackage{colortbl}

\renewcommand{\arraystretch}{1.3} 
\newcommand{\diag}{\mathop{\mathrm{diag}}}
\newcommand{\tabincell}[2]{\begin{tabular}{@{}#1@{}}#2\end{tabular}}  

\newtheorem{assumption}{\textbf{Assumption}}
\newtheorem{definition}{\textbf{Definition}}
\newtheorem{lemma}{\textbf{Lemma}}
\newtheorem{theorem}{\textbf{Theorem}}
\newtheorem{remark}{\textbf{Remark}}
\newtheorem{proposition}{\textbf{Proposition}}
\newtheorem{corollary}{\textbf{Corollary}}

\def\ba{\begin{array}}
	\def\ea{\end{array}}
\newcommand{\beq}{\begin{equation}}
\newcommand{\eeq}{\end{equation}}
\newcommand{\bq}{\begin{eqnarray}}
\newcommand{\eq}{\end{eqnarray}}
\newcommand{\bqn}{\begin{eqnarray*}}
	\newcommand{\eqn}{\end{eqnarray*}}
\newcommand{\bee}{\begin{enumerate}}
	\newcommand{\eee}{\end{enumerate}}
\newcommand{\bi}{\begin{itemize}}
	\newcommand{\ei}{\end{itemize}}
%\newcommand{\qed}{\hfill{$\blacksquare$}}
\newcommand{\ii}{\textbf{i}}

\usepackage{comment}
%\newboolean{showcomments}
%\setboolean{showcomments}{true}
\newcommand{\slow}[1]{\ifthenelse{\boolean{showcomments}}
	{ \textcolor{red}{(SL:  #1)}}{}}
\newcommand{\you}[1]{\ifthenelse{\boolean{showcomments}}
	{ \textcolor{green}{(PCY:  #1)}}{}}
\newcommand{\john}[1]{\ifthenelse{\boolean{showcomments}}
	{ \textcolor{blue}{(jp:  #1)}}{}}


% correct bad hyphenation here
\hyphenation{op-tical net-works semi-conduc-tor}


\begin{document}

\title{\Large On Power Networks Coupled with Market Dynamics}


\author{Pengcheng You \\  Nov 14 2017% <-this % stops a space
%\thanks{M. Shell was with the Department
%of Electrical and Computer Engineering, Georgia Institute of Technology, Atlanta,
%GA, 30332 USA e-mail: (see http://www.michaelshell.org/contact.html).}% <-this % stops a space
%\thanks{J. Doe and J. Doe are with Anonymous University.}% <-this % stops a space
%\thanks{Manuscript received April 19, 2005; revised August 26, 2015.}
}


% make the title area
\maketitle


%\begin{abstract}
%
%\end{abstract}

%
%\begin{IEEEkeywords}
%
%\end{IEEEkeywords}


\IEEEpeerreviewmaketitle


     

\textbf{Linear model}

Linear swing dynamics:
  \begin{subequations}
  	\begin{eqnarray}
  	\label{eq:nwdymvec:1}
  	\dot \theta 	& = &  \omega \\
  	\label{eq:nwdymvec:2}
  	M \dot \omega_\mathcal{G} & = & r_\mathcal{G} +p  - d_\mathcal{G}  -D_\mathcal{G}  \omega_\mathcal{G}  - C_{\mathcal{G}}B  \bar C^T \theta_{\mathcal{N}^+} \\
  	\label{eq:nwdymvec:3}
  	0 & = & r_{\mathcal{L}} - d_{\mathcal{L}}  -D_{\mathcal{L}} \omega_{\mathcal{L}} - C_{\mathcal{L}}B \bar C^T\theta_{\mathcal{N}^+} \\
  	\label{eq:nwdymvec:4}
    0 & = & r_0 -D_0 \omega_0 - C_0 B \bar C^T\theta_{\mathcal{N}^+} 
    \end{eqnarray}
    
Rational behavior of market participants:
    \begin{eqnarray}
  	\label{eq:nwdymvec:5}
  	T^p_j \dot p_j & =  & \lambda -  \omega_j   +  H_j \eta^- - H_j \eta^+ - J_j'(p_j)     \\
  	\label{eq:nwdymvec:6}
  	T^d_j \dot d_j & =  & U_j'(d_j) -  \lambda +  \omega_j - H_j \eta^- +  H_j\eta^+    
  	\end{eqnarray}\label{eq:nwdymvec}%
  \end{subequations}

Price dynamics:
\begin{subequations}
	\begin{eqnarray}
		\label{eq:ctr:1}
		\dot \lambda & = & \gamma^\lambda \left( - \mathbf 1^T_\mathcal{G} (r_\mathcal{G}+p-d_\mathcal{G})  - \mathbf 1^T_\mathcal{L} (r_\mathcal{L}-d_\mathcal{L}) -r_0 \right) \\
		\label{eq:ctr:2}
		\dot \eta^- & = & \Gamma^{\eta^-} \left[\underline{F} - H^T_{\mathcal{G}}(r_\mathcal{G}+p-d_\mathcal{G}) - H^T_\mathcal{L}(r_\mathcal{L}-d_\mathcal{L})  \right]^+_{\eta^-}   \\
		\label{eq:ctr:3}
		\dot \eta^+ & = & \Gamma^{\eta^-} \left[ H^T_{\mathcal{G}}(r_\mathcal{G}+p-d_\mathcal{G}) + H^T_\mathcal{L}(r_\mathcal{L}-d_\mathcal{L}) - \overline{F}  \right]^+_{\eta^+}   
	\end{eqnarray}\label{eq:ctr}%
\end{subequations}

\textbf{Nonlinear model}

Nonlinear swing dynamics:
\beq
\dot x= f(x)
\eeq
where $x:=(\theta,\omega,p,d,E)$

Rational behavior of market participants:
    \begin{eqnarray}
T^p_j \dot p_j & =  & u_j - J_j'(p_j)     \\
T^d_j \dot d_j & =  & U_j'(d_j) -  u_j
\end{eqnarray}%
where $u_j$ is the price at bus $j$.

Feedback price controller:
\beq
u= H x
\eeq

The problem is how to design $H$ to stabilize $x$, given the rational behavior of market participants.





\bibliographystyle{IEEEtran}  
\bibliography{bib}  


\end{document}


